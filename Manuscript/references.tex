\newpage
\makeatletter
\renewcommand\@biblabel[1]{\hspace*{-\labelsep}} % Remove the numbering
\makeatother
\setlength{\labelsep}{0pt} % Remove space where the label would be
\setlength{\itemindent}{-1.2em} % Adjust item indent
\setlength{\leftmargin}{1.2em} % Adjust left margin
\setlength{\itemsep}{1em} % Adjust the spacing between items
\begin{thebibliography}{99}
\setlength{\itemsep}{-0.025em}

\bibitem{} Abdu, R., Müller, C., Kirfel, L., \& Dackermann, T. (2025). Demonstrating understanding of proportionality through embodied interaction: Eye–hand coordination as evidence. \emph{ZDM–Mathematics Education, 57}(1), 89–106.

\bibitem{} Boaler, J. (1998). Open and closed mathematics: Student experiences and understandings. \emph{Journal for Research in Mathematics Education, 29}(1), 41–62.

\bibitem{} D'Mello, S., \& Graesser, A. (2012). Dynamics of affective states during complex learning. \emph{Learning and Instruction, 22}(2), 145–157.

\bibitem{} Fried, M. N. (2001). Can mathematics education and history of mathematics coexist? \emph{Science \& Education, 10}(4), 391–408.

\bibitem{} Gay, G. (2010). \emph{Culturally responsive teaching: Theory, research, and practice} (2nd ed.). Teachers College Press.

\bibitem{} Holmes, W., \& Bialik, M. (2023). Artificial intelligence in education: Promises and implications for teaching and learning. \emph{OECD Education Working Papers, No. 270}. OECD Publishing.

\bibitem{} Jahnke, H. N. (2000). The use of historical texts in mathematics education. \emph{Educational Studies in Mathematics, 41}(1), 63–87.

\bibitem{} Rexhepi, J., \& Makasevska, A. (2024). Impact of folding back instruction on third graders’ fraction understanding. \emph{Educational Studies in Mathematics, 115}(3), 483–502.

\bibitem{} Ryan, R. M., \& Deci, E. L. (2000). Self-determination theory and the facilitation of intrinsic motivation, social development, and well-being. \emph{American Psychologist, 55}(1), 68–78.

\bibitem{} Schoenfeld, A. H. (2004). The math wars. \emph{Educational Policy, 18}(1), 253–286.

\bibitem{} Sfard, A. (1991). On the dual nature of mathematical conceptions: Reflections on processes and objects as different sides of the same coin. \emph{Educational Studies in Mathematics, 22}(1), 1–36.

\bibitem{} Tall, D., \& Vinner, S. (1981). Concept image and concept definition in mathematics with particular reference to limits and continuity. \emph{Educational Studies in Mathematics, 12}(2), 151–169.

\bibitem{} Thompson, A. G. (1994). Teachers’ beliefs and conceptions: A synthesis of the research. In D. A. Grouws (Ed.), \emph{Handbook of Research on Mathematics Teaching and Learning} (pp. 127–146). Macmillan.

\bibitem{} Torrance, M., Lin, Y., \& Zhang, K. (2023). Artificial intelligence in STEM classrooms: Current practices and future possibilities. \emph{Journal of STEM Education, 24}(1), 13–28.

\bibitem{} Vygotsky, L. S. (1978). \emph{Mind in society: The development of higher psychological processes}. Harvard University Press.

\bibitem{} Wanzer, M. B., Frymier, A. B., \& Irwin, J. (2010). An explanation of the relationship between instructor humor and student learning: Instructional humor processing theory. \emph{Communication Education, 59}(1), 1–18.

\bibitem{} Zepeda, C. D., Richey, J. E., Ronevich, P., \& Nokes-Malach, T. J. (2015). Direct instruction of metacognition benefits adolescent science learning, transfer, and motivation: An in vivo study. \emph{Journal of Educational Psychology, 107}(4), 954–970.

\bibitem{} Zhai, X., Chu, H. E., \& Wang, J. (2023). Current trends and challenges in AI-based mathematics education. \emph{International Journal of Artificial Intelligence in Education, 33}(2), 345–366.

\end{thebibliography}
